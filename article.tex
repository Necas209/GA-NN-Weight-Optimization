\documentclass[a4paper, portuguese]{report}

\usepackage[ruled]{algorithm2e}
\usepackage{cite}
\usepackage{comment}
\usepackage{graphicx}
\usepackage{parskip}
\usepackage{minted}
\usepackage{titlesec}
\usepackage[portuguese]{babel}
\usepackage[hyphens]{url}
\usepackage[hidelinks,pdfusetitle]{hyperref}
\usepackage{xcolor}
\usepackage{siunitx}
\usepackage{textcmds}
\usepackage{neuralnetwork}
\usepackage[utf8]{inputenc}
\usepackage[T1]{fontenc}

\hypersetup{
    breaklinks=true,
    pdfauthor={Diogo Medeiros e João Santos},
    pdftitle={Otimização de Redes Neuronais usando Algoritmos Genéticos}
}
\urlstyle{same}

\providecommand{\keywords}[1]
{
  \small	
  \textbf{\textit{Palavras-chave --}} #1
}
 % define algorithm2e comments
\SetKwComment{Comment}{/* }{ */}

\setlength{\parindent}{2em}
\setlength{\parskip}{\baselineskip}
\titleformat{\chapter}[display]
   {\normalfont\huge\bfseries}{\chaptertitlename\ \thechapter}{10pt}{\Huge}
\titlespacing*{\chapter}{0pt}{0pt}{30pt}

\title{{\textbf{Computação Natural}}\\ Mestrado em Engenharia Informática - 1.º ano\\\vspace*{1cm} José Paulo Barroso de Moura Oliveira\\ Eduardo José Solteiro Pires\\\vspace*{3cm}\textbf{Trabalho Prático}\\\vspace*{0.5cm} Otimização de Redes Neuronais usando Algoritmos Genéticos \vspace*{1cm}}
\author{Diogo Medeiros (70633) \and João Santos (68843)}
\date{maio 2023}

\begin{document}

\begin{figure}
\includegraphics[width=8cm]{images/utad}
\label{fig:utad_logo}
\end{figure}

\maketitle

\begin{abstract}

Neste trabalho, será desenvolvido um sistema computacional baseado em agentes racionais, utilizando a ferramenta NetLogo, com o objetivo de simular a propagação de incêndios florestais, levando em consideração diversos fatores ambientais. Pretende-se entender melhor o comportamento do fogo em diferentes situações e determinar as melhores estratégias de combate e prevenção. Para alcançar esse objetivo, serão simulados quatro cenários distintos, variando a inclinação do terreno e a direção e intensidade do vento.

Através deste estudo, espera-se demonstrar a importância e complexidade da simulação de incêndios florestais e destacar o valor das abordagens baseadas em agentes para aprofundar a compreensão das interações e comportamentos envolvidos. O trabalho servirá como base para futuras investigações e melhorias no campo da prevenção e combate a incêndios florestais, procurando modelos mais refinados e eficazes que reduzam o impacto dos incêndios florestais nas pessoas e nos ecossistemas.

\keywords{simulação de incêndios florestais, agentes racionais, modelos baseados em agentes, NetLogo, fatores ambientais, topografia, direção e intensidade do vento, prevenção e combate a incêndios}

\end{abstract}

\tableofcontents
\listoffigures
\listoftables
\listofalgorithms

\chapter{Introdução}\label{chapter:introduction}
No âmbito da Unidade Curricular de Computação Natural, foi solicitado um trabalho prático que consiste no desenvolvimento de um algoritmo genético na linguagem de programação Python, com o objetivo de otimizar os parâmetros de uma Rede Neuronal Artificial.

\section{Algoritmos Genéticos}\label{sec:gen_algs}

Os algoritmos genéticos (GA) são algoritmos de pesquisa baseados na mecânica da seleção natural e nos princípios da genética, que combinam a sobrevivência dos indivíduos mais aptos numa população de entidades binárias - cromossomas, com a troca estruturada e aleatória de informação (Fig.~\ref{fig:ea_flowchart}). 

Em cada geração, um novo conjunto de indivíduos (sequências) é criado usando pedaços dos elementos mais aptos. Ainda que aleatórios, os  algoritmos genéticos não são uma simples \qq{caminhada} estocástica, explorando eficientemente a informação histórica para especular novos pontos de pesquisa com um desempenho esperado melhorado ~\cite{Goldberg1989GeneticLearning}.

\begin{figure}[htbp]
\centering
\begin{tikzpicture}[node distance = 2.9cm]
\tikzstyle{frame} = [rectangle, draw, text width=6em, text centered, minimum height=3em]
\tikzstyle{line} = [draw, ->]
\node [frame] (pop) {População};
\node [above=2cm, left of=pop] (init) {Inicialização};
\node [below=2cm, left of=pop] (term) {Finalização};
\node [frame, above=2cm, right of=pop] (parents)  {Pais};
\node [frame, below=2cm, right of=pop] (off)  {Filhos};
\path [line] (parents) -- node[right,align=left,pos=.5] {Cruzamento}(off);
\path [line] (init) |- (pop.170);
\path [line] (pop.190) -| (term);
\path [line] (off) -| node[below,pos=.25, align=center] {Mutação}(pop);
\path [line] (pop) |- node[above,pos=.75, align=center] {Seleção}(parents);
\end{tikzpicture}
\caption{Processo geral de um algoritmo genético}
\label{fig:ea_flowchart}
\end{figure}

\section{Redes Neuronais}\label{sec:neural_nets}

As Redes Neuronais Artificiais (ANNs) são modelos computacionais inspirados no funcionamento do cérebro humano, compostos por unidades de processamento, chamadas neurónios, responsáveis por receber entradas, processar informação e gerar saídas. As ANNs utilizam conexões sinápticas ajustáveis para aprender com os dados de entrada~\cite{Lippmann1988AnNets}.

Essas redes possuem a capacidade de aprender por meio de exemplos e generalizar esse conhecimento para novos dados. Por essa razão, elas são amplamente utilizadas em diversas áreas, como reconhecimento de padrões, classificação, previsão, processamento de imagens, entre outras~\cite{Jain1996ArtificialTutorial}.

Atualmente, as ANNs são consideradas ferramentas fundamentais para aprendizagem de máquina (ML) e inteligência artificial (AI), com inúmeras aplicações práticas em saúde, finanças, indústria e outros setores.

\section{Otimização de Redes Neuronais}\label{sec:optim_nns}

A crescente popularidade das ANNs na década de 80 potenciou o surgimento de novas técnicas para o seu treino e otimização, com particular atenção para os algoritmos genéticos. A retropropagação (BP), apesar da sua utilidade e simplicidade, apresentava limitações, entre as quais:
\begin{itemize}
    \item a pobre escalabilidade face ao aumento da complexidade do problema (maior dimensionalidade e/ou complexidade dos dados), resultando numa degradação de desempenho;
    \item a necessidade de diferenciabilidade no cálculo dos gradientes, que impede o seu uso em certos tipos de nós e critérios de otimização não-diferenciáveis.
\end{itemize}

Para superar essas limitações, Montana e Davis~\cite{Montana1989} propuseram um algoritmo genético capaz de codificar os pesos duma rede neuronal \textit{feed-forward} (FNN) numa lista de números reais, utilizando uma função de \textit{fitness} para avaliar o desempenho da rede, e incorporando operadores de mutação, cruzamento e gradiente para gerar novas soluções. 

Montana e Davis realizaram múltiplas experiências usando uma base de dados de imagens de energia acústica, e compararam o desempenho do algoritmo genético com o da retropropagação. Os resultados mostraram que o GA superou o BP em termos de velocidade de treino e capacidade de lidar com nós e critérios de otimização, provando os GAs como uma alternativa eficaz à retropropagação, especialmente em problemas complexos e não diferenciáveis.

\chapter{Problema}\label{chapter:problem}

O treino de redes neuronais é um processo fundamental para obter um desempenho adequado em tarefas de classificação, reconhecimento de padrões e outras aplicações. No entanto, encontrar os parâmetros ideais para uma rede neural pode ser um desafio devido à sua complexidade e ao vasto espaço de busca envolvido. 

Nesse contexto, técnicas como os algoritmos genéticos (GAs) têm mostrado eficácia na otimização dos parâmetros das redes neuronais. Os GAs combinam conceitos da teoria da evolução biológica, como a seleção natural, o cruzamento e a mutação, para explorar o espaço de pesquisa de forma eficiente. Estas técnicas permitem encontrar conjuntos de parâmetros que maximizem o desempenho da rede neuronal, melhorando a precisão das classificações e tornando-a mais adaptável a diferentes conjuntos de dados.

O objetivo do presente trabalho consiste em desenvolver um algoritmo genético capaz de encontrar os pesos ideais de uma rede neuronal, de modo a maximizar o seu desempenho na classificação de flores do género Iris \textit{L.}, recorrendo ao dataset Iris~\cite{Fisher1988}.

A rede neuronal em questão deverá ter a seguinte estrutura:
\begin{itemize}
    \item 4 neurónios na camada de entrada
    \item 10 neurónios na camada oculta
    \item 3 neurónios na camada de saída
\end{itemize}

O algoritmo desenvolvido deverá ser igualmente capaz de inibir conexões a certos neurónios da rede durante a sua execução. Essa modificação tem como objetivo explorar a capacidade da rede de desativar neurónios irrelevantes para a tarefa de classificação, visando otimizar a sua eficiência computacional.

\chapter{Metodologia}\label{chapter:methodology}
A escolha da linguagem Python para este trabalho deve-se a várias vantagens que a mesma oferece. Python é uma linguagem de programação de alto nível, reconhecida pela sua sintaxe simples e legibilidade, o que facilita o desenvolvimento e a manutenção do código. Por outro lado, Python possui uma ampla gama de bibliotecas e frameworks para aprendizagem de máquina e computação científica, como NumPy, scikit-learn e PyTorch, que tornam mais fácil a implementação de redes neuronais e algoritmos genéticos.

De modo a poder comparar os resultados obtidos, o algoritmo genético foi implementado de raiz e com recurso à biblioteca PyGAD \cite{Gad2021PyGAD:Library}. PyGAD é uma biblioteca open-source, compatível com Keras e PyTorch, utilizada para construir algoritmos genéticos e otimizar algoritmos de ML. Este oferece diferentes tipos de operadores de cruzamento, mutação e seleção, permitindo a otimização de inúmeros problemas, através da personalização da função de aptidão.

\section{Modelo de Classificação}\label{sec:neural_net}

Tal como foi referido em \ref{chapter:problem}, e respeitando as características do dataset IRIS, a rede neuronal tem 4 neurónios na camada de entrada, 10 neurónios na camada oculta, e 3 neurónios na camada de saída, a fim de classificar as 3 espécies.

\begin{figure}[htbp]
    \centering
        \begin{neuralnetwork}[height=5]
        \newcommand{\x}[2]{$x_#2$}
        \newcommand{\y}[2]{$y_#2$}
        \newcommand{\h}[2]{\ifnum #2=4 $\cdots$ \else {\ifnum #2>4 $h_{10}$ \else $h_#2$ \fi} \fi}
        \newcommand{\hfirst}[2]{\small $h^{(1)}_#2$}
        \inputlayer[count=4, bias=false, title=Camada\\de entrada, text=\x]
        \hiddenlayer[count=5, bias=false, title=Camada oculta, text=\h] \linklayers
        \outputlayer[count=3, title=Camada\\de saída, text=\y] \linklayers
    \end{neuralnetwork}
    \caption{Estrutura do modelo de classificação}
    \label{fig:neural_net}
\end{figure}

O modelo em questão é uma Fully Connected Neural Network (FCN), também conhecida por Rede Neuronal Totalmente Conectada, como se poder ver pela Fig.~\ref{fig:neural_net}. Trata-se de uma arquitetura de rede neuronal artificial na qual cada neurónio numa camada está conectado a todos os neurónios da camada seguinte - conexões totalmente conectadas ou conexões densas.

\begin{table}[htpb]
    \centering
    \begin{tabular}{cccc} \hline
    Camada & Pesos & Bias & Parâmetros \\ \hline
    Entrada (X) & 40 & 10 & 50 \\
    Oculta (H) & 30 & 3 & 33 \\
    Saída (Y) & - & - & - \\ \hline
    \textbf{Total} & 50 & 33 & 83 \\ \hline
    \end{tabular}
    \caption{Detalhes da rede neuronal}
    \label{tab:nn_summary}
\end{table}

Assim, e de acordo com a informação disponível na Tabela~\ref{tab:nn_summary}, o modelo em questão contém um total de 83 parâmetros que podem ser otimizados pelo algoritmo genético descrito na secção seguinte.

\section{Algoritmo Genético}\label{sec:gen_alg}

O objetivo principal deste algoritmo é fornecer uma ferramenta eficiente para modelar e prever o comportamento dos incêndios florestais, permitindo a avaliação dos efeitos de diferentes cenários na propagação do fogo. A simulação baseada em agentes é uma abordagem que considera a interação entre os elementos do ambiente e os agentes que nele atuam, permitindo a modelação de comportamentos complexos e dinâmicos. 

O algoritmo descrito aqui considera fatores como topografia, vento, temperatura, a estrutura da comunidade vegetal e outros elementos relevantes para o comportamento do fogo. A seguir, serão apresentados detalhes do funcionamento do algoritmo e como ele foi implementado para atender aos objetivos.

\SetKwFunction{calcAltitude}{calcAltitude}
\SetKwFunction{random}{random}
\SetKwFunction{plantTree}{plantTree}

\begin{algorithm}
    \caption{Criação da floresta (\texttt{createForest})}\label{alg:create_forest}
    $sparkFrequency \gets 150$\;
    $seed \gets forestSeed$\;
    \For{$patch \in patches$}{
        $altitude \gets \calcAltitude{pxcor}$\;
        $temperature \gets initialTemperature$\;
        \If{\random{$0, 100$} $<$ forestDensity}{
            \plantTree{pxcor, pycor}\;
        }
    }
\end{algorithm}

Em Alg.~\ref{alg:create_forest}, encontra-se detalhado o processo de população da floresta. Este começa por definir a frequência, em ticks, de criação de fagulhas, seguido da atribuição da \textit{seed} responsável pela distribuição de árvores, bem como pela criação do fogo. A cada célula é atribuída uma altitude dependente da sua posição, bem como uma temperatura inicial escolhida pelo utilizador. Também, segundo a densidade da floresta, são plantadas árvores nos diferentes patches.

\SetKwFunction{ignite}{ignite}
\SetKwFunction{saveConfig}{saveConfig}
\SetKwFunction{anyTreesBurning}{anyTreesBurning}
\SetKwFunction{saveIteration}{saveIteration}
\SetKwFunction{saveIterations}{saveIterations}
\SetKwFunction{fire}{fire}

\begin{algorithm}
    \caption{Criação do fogo inicial (\texttt{startFire})}\label{alg:start_fire}
    $patch \gets \random(patches)$\;
    \ignite{patch}\;
    \saveConfig{}\;
    $seed \gets \random{seeds}$\;
    \While{\anyTreesBurning{}}{
        \saveIteration{}\;
        \fire{}\;
    }
    \saveIterations{}\;
\end{algorithm}

Com a floresta plantada, resta criar o fogo inicial, tal como se pode ver em Alg.~\ref{alg:start_fire}. Começa-se por escolher aleatoriamente um patch, no qual é criada a primeira chama. De seguida, são salvas as configurações do cenário em ficheiro YAML, e reiniciada a \textit{seed}. O modelo entra num ciclo condicional, responsável por salvar o estado do modelo em cada iteração e de propagar o incêndio. Após o fim do ciclo, são guardados em ficheiro CSV os resultados das iterações.

\SetKwFunction{spreadFire}{spreadFire}
\SetKwFunction{canSpark}{canSpark}
\SetKwFunction{createSpark}{createSpark}
\SetKwFunction{forward}{forward}
\SetKwFunction{fadeEmbers}{fadeEmbers}
\SetKwFunction{tick}{tick}
\SetKwFunction{create}{create}

\begin{algorithm}
    \caption{Evolução do incêndio (\texttt{fire})}\label{alg:fire}
    \For{$tree \in trees$}{
        \If{isBurning}{
            \If{$color < ``yellow"$}{
                \spreadFire{neighbors, altitude}\;
            }
            \If{$color < ``brown"$}{
                \If{$\random{$0, 1$} < sparkProbability$ {\bf and} $ticksSinceSpark > sparkFrequency$}{
                    \create{spark}\;
                    $ticksSinceSpark \gets 0$\;
                }
                \Else{
                    $ticksSinceSpark \gets ticksSinceSpark + 1$\;
                }
            }
        }
    }
    \For{$spark \in sparks$}{
        \If{$position \neq finalPosition$}{
            \forward{$0.1$}\;
        }
        \Else{
            \ignite{patch-here}\;
        }
    }
    \fadeEmbers{}\;
    \tick{}\;
\end{algorithm}

O Algoritmo~\ref{alg:fire} representa o algoritmo responsável pela evolução do incêndio ao longo da execução do modelo. Em cada tick, as árvores a arder espalham o fogo para as células vizinhas e geram fagulhas, segundo uma certa probabilidade, de acordo com a cor das suas folhas - esta representa o estado da árvore, mais ou menos queimada. Já as fagulhas movem-se em direção à sua posição final, na qual incendeiam as árvores presentes. Por fim, o estado de queima das árvores é atualizado com o procedimento \texttt{fadeEmbers}, responsável por alterar as propriedades da árvore, em particular a sua cor.

\begin{algorithm}
    \caption{Ignição do fogo (\texttt{ignite})}\label{alg:ignite}
    \create{fire}\;
    \For{$tree \in trees{-}here$}{
        \If{{\bf not} ($isBurning$ {\bf and} $isBurnt$)}{
            $isBurning \gets true$\;
        }
    }
\end{algorithm}

Para incendiar as árvores, recorremos ao procedimento \texttt{ignite}, definido em Alg.~\ref{alg:ignite}. Este algoritmo é responsável por criar um agente \textit{fire} na célula atual, bem como de incendiar quaisquer árvores que não estejam a arder na mesma.

\SetKwFunction{towards}{towards}
\SetKwFunction{meanTemperature}{meanTemperature}

\begin{algorithm}
    \caption{Propagação do fogo (\texttt{spreadFire})}\label{alg:spread_fire}
    \KwIn{fireAltitude}
    $probability \gets spreadProbability$\;
    $direction \gets \towards{this}$\;
    \Switch{direction}{
        \Case{\ang{0}}{
            $probability \gets probability - northWindSpeed$\;
        }
        \Case{\ang{90}}{
            $probability \gets probability - eastWindSpeed$\;
        }
        \Case{\ang{180}}{
            $probability \gets probability + northWindSpeed$\;
        }
        \Case{\ang{270}}{
            $probability \gets probability + eastWindSpeed$\;
        }
    }
    $meanTemp \gets \meanTemperature{neighbors}$\;
    $probability \gets probability + \ln^2{(meanTemp + 1)}$\;
    \If{$fireAltitude > altitude$}{
        $probability \gets probability\cdot(1+|\tan(\frac{inclination}{3})|)$\;
    }
    \If{\random{$0, 100$} $<$ probability}{
        \ignite{patch-here}\;
    }
\end{algorithm}

Por fim, temos o algoritmo responsável pela propagação do fogo em si, retratado em Alg.~\ref{alg:spread_fire}. Começa-se por determinar a direção do fogo em relação à célula atual, e por inicializar a probabilidade de propagação com o valor definido pelo utilizador no início do programa. De seguida, consoante a direção do vento, ajusta-se adequadamente esta, tal que o mesmo possa contribuir, positiva ou negativamente para essa propagação. Segue-se o cálculo da temperatura média nas células vizinhas, que é igualmente usada para modificar novamente a propriedade (quanto mais quente estiver o ambiente em redor, melhor se propaga o fogo). Por fim, e porque o fogo tende a subir, quando enfrenta terrenos íngremes, a probabilidade é alterada uma última vez para refletir a inclinação do terreno. Termina-se por incendiar a célula presente, segundo o valor final da probabilidade de propagação.

Todos os detalhes da implementação do modelo de simulação em NetLogo, bem como do módulo desenvolvido em Python para o processamento dos resultados, podem ser consultados no Apêndice~\ref{chapter:appendix}.

\chapter{Resultados}\label{chapter:results}
Neste capítulo serão apresentados e analisados os resultados da simulação de um incêndio florestal, em quatro cenários distintos. Estes cenários apresentam configurações de modelo semelhantes, diferindo apenas nos seguintes parâmetros: velocidade do vento no sentido Norte e Este, e inclinação do terreno. Todos os valores podem ser consultados na Tab.~\ref{tab:scenarios}.

\begin{table}[tbhp]
    \centering
    \begin{tabular}{c|cccc} 
    \hline
    \textbf{Propriedades} & \textbf{Cenário 1} & \textbf{Cenário 2} & \textbf{Cenário 3} & \textbf{Cenário 4} \\ 
    \hline
    Densidade & 25\% & - & - & - \\
    Carvalhos/pinheiros & 271/272 & - & - & - \\
    Prob. propagação & 0.4 & - & - & - \\
    Temperatura inicial & $\SI{20}{\degreeCelsius}$ & - & - & - \\
    Inclinação & \ang{30} & \ang{-30} & \ang{30} & \ang{-30} \\
    Vento Norte & -10 & -10 & 10 & 10 \\
    Vento Este & 15 & 15 & 15 & 15 \\
    \hline
    \end{tabular}
    \caption{Configurações dos diferentes cenários}
    \label{tab:scenarios}
\end{table}

Cada cenário foi executado onze vezes, e para cada execução foram com guardados em CSV os seguintes parâmetros do modelo: tick, número de árvores ardidas - absoluto, percentual e para cada tipo de árvore, número de fagulhas na floresta, temperatura média e máxima.

\section{Cenário 1}\label{sec:scenario1}

O primeiro cenário apresenta uma inclinação positiva do terreno, de \ang{30}, com vento sudeste de intensidade $(-10, 15)$ \si{\meter\per\second}, ou seja, a favor da propagação do incêndio. O modelo terminou com 71.3\% da floresta ardida, após 3377 ticks, ou 387 árvores em termos absolutos - 205 carvalhos e 182 pinheiros.

De início, o fogo propaga-se rápido, como se pode notar pelas curvas acentuadas das Fig.~\ref{fig:S1ForestBurnt} e \ref{fig:S1TreesBurnt}, acabando por abrandar por volta dos 2000 ticks e estabilizar pouco tempo depois. As fagulhas, dado a sua aleatoriedade e condições específicas de geração, apresentam uma evolução mais errática, evidente pela Fig.~\ref{fig:S1Sparks}, só começando a espalhar-se a partir dos 400 ticks, e atingindo um pico de 15 fagulhas por tick por volta dos 1500 ticks. Dado que a floresta é pouco densa, a temperatura média não ultrapassou os $\SI{330}{\degreeCelsius}$, já a temperatura máxima, atingida após 800 ticks (Fig.~\ref{fig:S1Temp}), foi muito superior, cerca de $\SI{1870}{\degreeCelsius}$.
    
\begin{figure}[H]
    \includegraphics[width=\linewidth]{imagens/scenario1/forest_fire.png}
    \caption{Cenário 1 - Total de floresta ardida (em percentagem)}
    \label{fig:S1ForestBurnt}        
\end{figure}
    
\begin{figure}[H]
    \centering
    \includegraphics[width=\linewidth]{imagens/scenario1/trees_burnt.png}
    \caption{Cenário 1 - Total de árvores ardidas}
    \label{fig:S1TreesBurnt}
\end{figure}

\begin{figure}[H]
    \centering
    \includegraphics[width=\textwidth]{imagens/scenario1/sparks.png}
    \caption{Cenário 1 - Evolução do número de fagulhas}
    \label{fig:S1Sparks}
\end{figure}

\begin{figure}[H]
    \centering
    \includegraphics[width=\textwidth]{imagens/scenario1/temperature.png}
    \caption{Cenário 1 - Evolução da temperatura média e máxima}
    \label{fig:S1Temp}
\end{figure}

\section{Cenário 2}\label{sec:scenario2}

O segundo cenário apresenta uma inclinação negativa do terreno, de \ang{-30}, com vento sudeste de intensidade $(-10, 15)$ \si{\meter\per\second}, ou seja, contra a propagação do incêndio. O modelo terminou com 71.4\% da floresta ardida, após 3359 ticks, ou 388 árvores em termos absolutos - 206 carvalhos e 182 pinheiros.

Apesar das condições adversas, o fogo propagou-se rapidamente, à semelhança do que aconteceu em \ref{sec:scenario1}. Quando analisadas as curvas das Fig.~\ref{fig:S2ForestBurnt} e \ref{fig:S2TreesBurnt}, torna-se evidente que o modelo teve uma evolução quase idêntica, porventura graças à alta probabilidade de propagação (40\%). Novamente, as fagulhas, apresentam uma evolução mais instável, evidente pela Fig.~\ref{fig:S2Sparks}, atingindo um pico de 16 fagulhas por tick por volta dos 1400 ticks. Por fim, a temperatura média ficou-se igualmente pelos $\SI{330}{\degreeCelsius}$, já a temperatura máxima, atingida após 800 ticks (Fig.~\ref{fig:S1Temp}), foi à volta de $\SI{1870}{\degreeCelsius}$.

\begin{figure}[H]
    \centering
    \includegraphics[width=\textwidth]{imagens/scenario2/forest_fire.png}
    \caption{Cenário 2 - Total de floresta ardida (em percentagem)}
    \label{fig:S2ForestBurnt}
\end{figure}

\begin{figure}[H]
    \centering
    \includegraphics[width=\textwidth]{imagens/scenario2/trees_burnt.png}
    \caption{Cenário 2 - Total de árvores ardidas}
    \label{fig:S2TreesBurnt}
\end{figure}

\begin{figure}[H]
    \centering
    \includegraphics[width=\textwidth]{imagens/scenario2/sparks.png}
    \caption{Cenário 2 - Evolução do número de fagulhas}
    \label{fig:S2Sparks}
\end{figure}

\begin{figure}[H]
    \centering
    \includegraphics[width=\textwidth]{imagens/scenario2/temperature.png}
    \caption{Cenário 2 - Evolução da temperatura média e máxima}
    \label{fig:S2Temp}
\end{figure}

\section{Cenário 3}\label{sec:scenario3}

O terceiro cenário apresenta uma inclinação positiva do terreno, de \ang{30}, com vento noroeste de intensidade $(10, -15)$ \si{\meter\per\second}, ou seja, contra da propagação do incêndio. O modelo terminou com 93.4\% da floresta ardida, após 3042 ticks, ou 508 árvores em termos absolutos - 250 carvalhos e 258 pinheiros. 

O incêndio foi capaz de destruir uma maior área da floresta, visível em Fig.~\ref{fig:S3ForestBurnt}, em menos tempo que os cenários anteriores. Ao nível do tipo de árvores ardidas, dada a sua idêntica distribuição pelo terreno, não houve variações significativas entre pinheiros e carvalhos, como se pode constatar pela Fig.~\ref{fig:S3TreesBurnt}. Também ao nível das fagulhas criadas, é possível notar um aumento considerável, com o modelo a atingir um pico de 18.5 fagulhas por tick por volta 
1500 ticks (Fig.~\ref{fig:S3Sparks}). A temperatura média final foi ligeiramente superior, na ordem dos $\SI{412}{\degreeCelsius}$, já a temperatura máxima, atingida após 800 ticks (Fig.~\ref{fig:S3Temp}), manteve-se nos $\SI{1870}{\degreeCelsius}$.

\begin{figure}[H]
    \centering
    \includegraphics[width=\textwidth]{imagens/scenario3/forest_fire.png}
    \caption{Cenário 3 - Total de floresta ardida (em percentagem)}
    \label{fig:S3ForestBurnt}
\end{figure}

\begin{figure}[H]
    \centering
    \includegraphics[width=\textwidth]{imagens/scenario3/trees_burnt.png}
    \caption{Cenário 3 - Total de árvores ardidas}
    \label{fig:S3TreesBurnt}
\end{figure}

\begin{figure}[H]
    \centering
    \includegraphics[width=\textwidth]{imagens/scenario3/sparks.png}
    \caption{Cenário 3 - Evolução do número de fagulhas}
    \label{fig:S3Sparks}
\end{figure}

\begin{figure}[H]
    \centering
    \includegraphics[width=\textwidth]{imagens/scenario3/temperature.png}
    \caption{Cenário 3 - Evolução da temperatura média e máxima}
    \label{fig:S3Temp}
\end{figure}

\section{Cenário 4}\label{sec:scenario4}

O quarto e último cenário apresenta uma inclinação negativa do terreno, de \ang{-30}, com vento noroeste de intensidade $(10, -15)$ \si{\meter\per\second}, ou seja, a favor da propagação do incêndio. O modelo terminou com 93.3\% da floresta ardida, após 3103 ticks, ou 507 árvores em termos absolutos - 250 carvalhos e 257 pinheiros.

Este cenário apresenta condições de vento idênticas ao descrito em \ref{sec:scenario3}, pelo que o modelo em si também evidenciou uma evolução do incêndio similar, quer ao nível da floresta ardida (Fig.~\ref{fig:S4ForestBurnt}), quer ao nível do tipo de árvores ardidas (Fig.~\ref{fig:S4TreesBurnt}). No que concerne as fagulhas, o modelo atingiu um pico marginalmente superior, de 18.5 fagulhas por tick por volta 
1500 ticks (Fig.~\ref{fig:S4Sparks}). A temperatura média final foi por volta de $\SI{411}{\degreeCelsius}$, já a temperatura máxima, atingida após 800 ticks (Fig.~\ref{fig:S3Temp}), manteve-se nos $\SI{1870}{\degreeCelsius}$.

\begin{figure}[H]
    \centering
    \includegraphics[width=\textwidth]{imagens/scenario4/forest_fire.png}
    \caption{Cenário 4 - Total de floresta ardida (em percentagem)}
    \label{fig:S4ForestBurnt}
\end{figure}

\begin{figure}[H]
    \centering
    \includegraphics[width=\textwidth]{imagens/scenario4/trees_burnt.png}
    \caption{Cenário 4 - Total de árvores ardidas}
    \label{fig:S4TreesBurnt}
\end{figure}

\begin{figure}[H]
    \centering
    \includegraphics[width=\textwidth]{imagens/scenario4/sparks.png}
    \caption{Cenário 4 - Evolução do número de fagulhas}
    \label{fig:S4Sparks}
\end{figure}

\begin{figure}[H]
    \centering
    \includegraphics[width=\textwidth]{imagens/scenario4/temperature.png}
    \caption{Cenário 4 - Evolução da temperatura média e máxima}
    \label{fig:S4Temp}
\end{figure}

\chapter{Conclusões}
Neste trabalho, desenvolveu-se um sistema computacional com agentes racionais utilizando a ferramenta NetLogo, com o objetivo de simular a propagação de um incêndio florestal considerando diferentes fatores ambientais. Foram analisados quatro cenários distintos, variando a inclinação do terreno e a direção do vento.

Os resultados obtidos não foram conforme o esperado, uma vez que os cenários com vento a favor e contra a inclinação apresentaram valores contraditórios. As nossas expectativas eram de que a propagação do incêndio seria maior nos casos em que o vento estivesse a favor da inclinação, e menor nos casos em que o vento estivesse contra a inclinação. No entanto, o que se observou foi que a percentagem da floresta ardida e outros parâmetros analisados não seguiram essa lógica.

Uma possível explicação para os resultados contraditórios pode ser a simplificação de alguns elementos do modelo, bem como a aleatoriedade de fatores como a geração de fagulhas. Outro fator que poderá ter contribuído para essa contradição é a alta probabilidade de propagação utilizada, que pode ter mascarado o efeito esperado da direção do vento e da inclinação do terreno.

A partir das conclusões deste trabalho, é possível sugerir melhorias no modelo de simulação para investigação futura. Estas melhorias podem incluir a consideração de outros fatores ambientais, como a humidade do ar e do solo, e uma análise mais aprofundada das probabilidades de propagação. Além disso, a realização de mais testes com diferentes configurações de parâmetros pode ajudar a compreender melhor as discrepâncias observadas.

Concluímos que a simulação baseada em agentes é uma ferramenta valiosa no estudo de incêndios florestais, permitindo aos investigadores compreender melhor as complexas interações e comportamentos envolvidos. Essas simulações têm sido usadas para explorar tópicos como estratégias de combate a incêndios, impacto do clima e topografia, e fatores humanos na gestão de incêndios. A melhor compreensão destes fatores pode levar a decisões mais informadas e práticas mais eficazes, tornando a simulação baseada em agentes uma ferramenta essencial na redução do impacto dos incêndios florestais. Embora os resultados não tenham sido os esperados, o trabalho realizado neste projeto servirá como base para futuras investigações e melhorias no campo da prevenção e combate a incêndios florestais.
\appendix
\chapter{Anexos dos ficheiros fonte}\label{chapter:appendix}
\section{Especificação do algoritmo genético}\label{sec:model_spec}
\inputminted[breaklines]{python}{src/genalg.py}

\bibliographystyle{IEEEtran}
\bibliography{article/references}

\end{document}

