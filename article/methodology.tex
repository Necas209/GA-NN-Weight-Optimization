A escolha da linguagem Python para este trabalho deve-se a várias vantagens que a mesma oferece. Python é uma linguagem de programação de alto nível, reconhecida pela sua sintaxe simples e legibilidade, o que facilita o desenvolvimento e a manutenção do código. Por outro lado, Python possui uma ampla gama de bibliotecas e frameworks para aprendizagem de máquina e computação científica, como NumPy, scikit-learn e PyTorch, que tornam mais fácil a implementação de redes neuronais e algoritmos genéticos.

De modo a poder comparar os resultados obtidos, o algoritmo genético foi implementado de raiz e com recurso à biblioteca PyGAD \cite{Gad2021PyGAD:Library}. PyGAD é uma biblioteca open-source, compatível com Keras e PyTorch, utilizada para construir algoritmos genéticos e otimizar algoritmos de ML. Este oferece diferentes tipos de operadores de cruzamento, mutação e seleção, permitindo a otimização de inúmeros problemas, através da personalização da função de aptidão.

\section{Modelo de Classificação}\label{sec:neural_net}

Tal como foi referido em \ref{chapter:problem}, e respeitando as características do dataset IRIS, a rede neuronal tem 4 neurónios na camada de entrada, 10 neurónios na camada oculta, e 3 neurónios na camada de saída, a fim de classificar as 3 espécies.

\begin{figure}[htbp]
    \centering
        \begin{neuralnetwork}[height=5]
        \newcommand{\x}[2]{$x_#2$}
        \newcommand{\y}[2]{$y_#2$}
        \newcommand{\h}[2]{\ifnum #2=4 $\cdots$ \else {\ifnum #2>4 $h_{10}$ \else $h_#2$ \fi} \fi}
        \newcommand{\hfirst}[2]{\small $h^{(1)}_#2$}
        \inputlayer[count=4, bias=false, title=Camada\\de entrada, text=\x]
        \hiddenlayer[count=5, bias=false, title=Camada oculta, text=\h] \linklayers
        \outputlayer[count=3, title=Camada\\de saída, text=\y] \linklayers
    \end{neuralnetwork}
    \caption{Estrutura do modelo de classificação}
    \label{fig:neural_net}
\end{figure}

O modelo em questão é uma Fully Connected Neural Network (FCN), também conhecida por Rede Neuronal Totalmente Conectada, como se poder ver pela Fig.~\ref{fig:neural_net}. Trata-se de uma arquitetura de rede neuronal artificial na qual cada neurónio numa camada está conectado a todos os neurónios da camada seguinte - conexões totalmente conectadas ou conexões densas.

\begin{table}[htpb]
    \centering
    \begin{tabular}{cccc} \hline
    Camada & Pesos & Bias & Parâmetros \\ \hline
    Entrada (X) & 40 & 10 & 50 \\
    Oculta (H) & 30 & 3 & 33 \\
    Saída (Y) & - & - & - \\ \hline
    \textbf{Total} & 50 & 33 & 83 \\ \hline
    \end{tabular}
    \caption{Detalhes da rede neuronal}
    \label{tab:nn_summary}
\end{table}

Assim, e de acordo com a informação disponível na Tabela~\ref{tab:nn_summary}, o modelo em questão contém um total de 83 parâmetros que podem ser otimizados pelo algoritmo genético descrito na secção seguinte.

\section{Algoritmo Genético}\label{sec:gen_alg}

O objetivo principal deste algoritmo é fornecer uma ferramenta eficiente para modelar e prever o comportamento dos incêndios florestais, permitindo a avaliação dos efeitos de diferentes cenários na propagação do fogo. A simulação baseada em agentes é uma abordagem que considera a interação entre os elementos do ambiente e os agentes que nele atuam, permitindo a modelação de comportamentos complexos e dinâmicos. 

O algoritmo descrito aqui considera fatores como topografia, vento, temperatura, a estrutura da comunidade vegetal e outros elementos relevantes para o comportamento do fogo. A seguir, serão apresentados detalhes do funcionamento do algoritmo e como ele foi implementado para atender aos objetivos.

\SetKwFunction{calcAltitude}{calcAltitude}
\SetKwFunction{random}{random}
\SetKwFunction{plantTree}{plantTree}

\begin{algorithm}
    \caption{Criação da floresta (\texttt{createForest})}\label{alg:create_forest}
    $sparkFrequency \gets 150$\;
    $seed \gets forestSeed$\;
    \For{$patch \in patches$}{
        $altitude \gets \calcAltitude{pxcor}$\;
        $temperature \gets initialTemperature$\;
        \If{\random{$0, 100$} $<$ forestDensity}{
            \plantTree{pxcor, pycor}\;
        }
    }
\end{algorithm}

Em Alg.~\ref{alg:create_forest}, encontra-se detalhado o processo de população da floresta. Este começa por definir a frequência, em ticks, de criação de fagulhas, seguido da atribuição da \textit{seed} responsável pela distribuição de árvores, bem como pela criação do fogo. A cada célula é atribuída uma altitude dependente da sua posição, bem como uma temperatura inicial escolhida pelo utilizador. Também, segundo a densidade da floresta, são plantadas árvores nos diferentes patches.

\SetKwFunction{ignite}{ignite}
\SetKwFunction{saveConfig}{saveConfig}
\SetKwFunction{anyTreesBurning}{anyTreesBurning}
\SetKwFunction{saveIteration}{saveIteration}
\SetKwFunction{saveIterations}{saveIterations}
\SetKwFunction{fire}{fire}

\begin{algorithm}
    \caption{Criação do fogo inicial (\texttt{startFire})}\label{alg:start_fire}
    $patch \gets \random(patches)$\;
    \ignite{patch}\;
    \saveConfig{}\;
    $seed \gets \random{seeds}$\;
    \While{\anyTreesBurning{}}{
        \saveIteration{}\;
        \fire{}\;
    }
    \saveIterations{}\;
\end{algorithm}

Com a floresta plantada, resta criar o fogo inicial, tal como se pode ver em Alg.~\ref{alg:start_fire}. Começa-se por escolher aleatoriamente um patch, no qual é criada a primeira chama. De seguida, são salvas as configurações do cenário em ficheiro YAML, e reiniciada a \textit{seed}. O modelo entra num ciclo condicional, responsável por salvar o estado do modelo em cada iteração e de propagar o incêndio. Após o fim do ciclo, são guardados em ficheiro CSV os resultados das iterações.

\SetKwFunction{spreadFire}{spreadFire}
\SetKwFunction{canSpark}{canSpark}
\SetKwFunction{createSpark}{createSpark}
\SetKwFunction{forward}{forward}
\SetKwFunction{fadeEmbers}{fadeEmbers}
\SetKwFunction{tick}{tick}
\SetKwFunction{create}{create}

\begin{algorithm}
    \caption{Evolução do incêndio (\texttt{fire})}\label{alg:fire}
    \For{$tree \in trees$}{
        \If{isBurning}{
            \If{$color < ``yellow"$}{
                \spreadFire{neighbors, altitude}\;
            }
            \If{$color < ``brown"$}{
                \If{$\random{$0, 1$} < sparkProbability$ {\bf and} $ticksSinceSpark > sparkFrequency$}{
                    \create{spark}\;
                    $ticksSinceSpark \gets 0$\;
                }
                \Else{
                    $ticksSinceSpark \gets ticksSinceSpark + 1$\;
                }
            }
        }
    }
    \For{$spark \in sparks$}{
        \If{$position \neq finalPosition$}{
            \forward{$0.1$}\;
        }
        \Else{
            \ignite{patch-here}\;
        }
    }
    \fadeEmbers{}\;
    \tick{}\;
\end{algorithm}

O Algoritmo~\ref{alg:fire} representa o algoritmo responsável pela evolução do incêndio ao longo da execução do modelo. Em cada tick, as árvores a arder espalham o fogo para as células vizinhas e geram fagulhas, segundo uma certa probabilidade, de acordo com a cor das suas folhas - esta representa o estado da árvore, mais ou menos queimada. Já as fagulhas movem-se em direção à sua posição final, na qual incendeiam as árvores presentes. Por fim, o estado de queima das árvores é atualizado com o procedimento \texttt{fadeEmbers}, responsável por alterar as propriedades da árvore, em particular a sua cor.

\begin{algorithm}
    \caption{Ignição do fogo (\texttt{ignite})}\label{alg:ignite}
    \create{fire}\;
    \For{$tree \in trees{-}here$}{
        \If{{\bf not} ($isBurning$ {\bf and} $isBurnt$)}{
            $isBurning \gets true$\;
        }
    }
\end{algorithm}

Para incendiar as árvores, recorremos ao procedimento \texttt{ignite}, definido em Alg.~\ref{alg:ignite}. Este algoritmo é responsável por criar um agente \textit{fire} na célula atual, bem como de incendiar quaisquer árvores que não estejam a arder na mesma.

\SetKwFunction{towards}{towards}
\SetKwFunction{meanTemperature}{meanTemperature}

\begin{algorithm}
    \caption{Propagação do fogo (\texttt{spreadFire})}\label{alg:spread_fire}
    \KwIn{fireAltitude}
    $probability \gets spreadProbability$\;
    $direction \gets \towards{this}$\;
    \Switch{direction}{
        \Case{\ang{0}}{
            $probability \gets probability - northWindSpeed$\;
        }
        \Case{\ang{90}}{
            $probability \gets probability - eastWindSpeed$\;
        }
        \Case{\ang{180}}{
            $probability \gets probability + northWindSpeed$\;
        }
        \Case{\ang{270}}{
            $probability \gets probability + eastWindSpeed$\;
        }
    }
    $meanTemp \gets \meanTemperature{neighbors}$\;
    $probability \gets probability + \ln^2{(meanTemp + 1)}$\;
    \If{$fireAltitude > altitude$}{
        $probability \gets probability\cdot(1+|\tan(\frac{inclination}{3})|)$\;
    }
    \If{\random{$0, 100$} $<$ probability}{
        \ignite{patch-here}\;
    }
\end{algorithm}

Por fim, temos o algoritmo responsável pela propagação do fogo em si, retratado em Alg.~\ref{alg:spread_fire}. Começa-se por determinar a direção do fogo em relação à célula atual, e por inicializar a probabilidade de propagação com o valor definido pelo utilizador no início do programa. De seguida, consoante a direção do vento, ajusta-se adequadamente esta, tal que o mesmo possa contribuir, positiva ou negativamente para essa propagação. Segue-se o cálculo da temperatura média nas células vizinhas, que é igualmente usada para modificar novamente a propriedade (quanto mais quente estiver o ambiente em redor, melhor se propaga o fogo). Por fim, e porque o fogo tende a subir, quando enfrenta terrenos íngremes, a probabilidade é alterada uma última vez para refletir a inclinação do terreno. Termina-se por incendiar a célula presente, segundo o valor final da probabilidade de propagação.

Todos os detalhes da implementação do modelo de simulação em NetLogo, bem como do módulo desenvolvido em Python para o processamento dos resultados, podem ser consultados no Apêndice~\ref{chapter:appendix}.