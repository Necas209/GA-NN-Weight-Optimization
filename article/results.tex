Neste capítulo serão apresentados e analisados os resultados da execução do algorithmo genético, descrito na Sec.~\ref{sec:gen_alg}, para a otimização do modelo de classificação, descrito na Sec.~\ref{sec:neural_net}, bem como os resultados obtidos com a biblioteca PyGAD\@.

Com o auxílio da biblioteca \texttt{scikit-learn}, foi possível carregar o dataset IRIS usando a função \texttt{load\_iris}, normalizá-lo com a classe \texttt{StandardScaler}, e dividi-lo em conjuntos de treino e teste, com a função \texttt{train\_test\_split}, segundo a proporção 90/10.
Após a divisão, os conjuntos de treino e teste foram convertidos em \texttt{Tensors} do PyTorch, de modo a poderem ser utilizados pelo modelo de classificação.

A seguir, foi criado um modelo de classificação, com a estrutura descrita na Sec.~\ref{sec:neural_net}, com a função de ativação \texttt{ReLU} na camada oculta,
e definido o critério de erro como \texttt{CrossEntropyLoss}, dado se tratar de um problema de classificação multiclasse. 

Foram definidas as funções \texttt{load\_params}, responsável por atualizar os pesos do modelo com os valores passados como argumento, a função de aptidão, que recebe como argumento um vetor de parâmetros, e retorna o inverso da função de erro, calculada com base no conjunto de treino, e a função de \textit{on\_generation}, chamada a cada geração, que imprime os valores da melhor, pior e aptidão média da população.

Por fim, o algoritmo foi instanciado com os os parâmetros apresentados na Tabela~\ref{tab:ga_param_values}, que também contém os valores utilizados na biblioteca PyGAD\@.
\begin{table}[htbp]
    \centering
    \begin{tabular}{ccc}
        \hline
        \textbf{Parâmetro} & \textbf{GeneticAlgorithm} & \textbf{PyGAD} \\
        \hline
        Tamanho da população & 30 & 10 \\
        Número de gerações & 600 & 200 \\
        Intervalo de gerações & 50 & - \\
        Probabilidade de mutação & 0.05 & - \\
        Probabilidade de \textit{crossover} & 0.95 & - \\
        Probabilidade de desconexão & $10^{-4}$ & - \\
        Elitismo & \texttt{True} & \texttt{True} \\
        \hline
    \end{tabular}
    \caption{Valores dos parâmetros do algoritmo genético}
    \label{tab:ga_param_values}
\end{table}