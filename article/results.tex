Neste capítulo serão apresentados e analisados os resultados da simulação de um incêndio florestal, em quatro cenários distintos.
Estes cenários apresentam configurações de modelo semelhantes, diferindo apenas nos seguintes parâmetros: velocidade do vento no sentido norte e este, e inclinação do terreno.
Todos os valores podem ser consultados na Tab.~\ref{tab:scenarios}.

\begin{table}[tbhp]
    \centering
    \begin{tabular}{c|cccc}
    \hline
    \textbf{Propriedades} & \textbf{Cenário 1} & \textbf{Cenário 2} & \textbf{Cenário 3} & \textbf{Cenário 4} \\
    \hline
    Densidade & 25\% & - & - & - \\
    Carvalhos/pinheiros & 271/272 & - & - & - \\
    Prob.
    propagação & 0.4 & - & - & - \\
    Temperatura inicial & $\SI{20}{\degreeCelsius}$ & - & - & - \\
    Inclinação & \ang{30} & \ang{-30} & \ang{30} & \ang{-30} \\
    Vento Norte & -10 & -10 & 10 & 10 \\
    Vento Este & 15 & 15 & 15 & 15 \\
    \hline
    \end{tabular}
    \caption{Configurações dos diferentes cenários}
    \label{tab:scenarios}
\end{table}

Cada cenário foi executado onze vezes, e para cada execução foram com guardados em CSV os seguintes parâmetros do modelo: tick, número de árvores ardidas - absoluto, percentual e para cada tipo de árvore, número de fagulhas na floresta, temperatura média e máxima.

\section{Cenário 1}\label{sec:scenario1}

O primeiro cenário apresenta uma inclinação positiva do terreno, de \ang{30}, com vento sudeste de intensidade $(-10, 15)$ \si{\meter\per\second}, ou seja, a favor da propagação do incêndio.
O modelo terminou com 71.3\% da floresta ardida, após 3377 ticks, ou 387 árvores em termos absolutos - 205 carvalhos e 182 pinheiros.

De início, o fogo propaga-se rápido, como se pode notar pelas curvas acentuadas das Fig.~\ref{fig:S1ForestBurnt} e~\ref{fig:S1TreesBurnt}, acabando por abrandar por volta dos 2000 ticks e estabilizar pouco tempo depois.
As fagulhas, dado a sua aleatoriedade e condições específicas de geração, apresentam uma evolução mais errática, evidente pela Fig.~\ref{fig:S1Sparks}, só começando a espalhar-se a partir dos 400 ticks, e atingindo um pico de 15 fagulhas por tick por volta dos 1500 ticks.
Dado que a floresta é pouco densa, a temperatura média não ultrapassou os $\SI{330}{\degreeCelsius}$, já a temperatura máxima, atingida após 800 ticks (Fig.~\ref{fig:S1Temp}), foi muito superior, cerca de $\SI{1870}{\degreeCelsius}$.