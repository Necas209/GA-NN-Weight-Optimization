Neste trabalho, desenvolveu-se um sistema computacional com agentes racionais utilizando a ferramenta NetLogo, com o objetivo de simular a propagação de um incêndio florestal considerando diferentes fatores ambientais. Foram analisados quatro cenários distintos, variando a inclinação do terreno e a direção do vento.

Os resultados obtidos não foram conforme o esperado, uma vez que os cenários com vento a favor e contra a inclinação apresentaram valores contraditórios. As nossas expectativas eram de que a propagação do incêndio seria maior nos casos em que o vento estivesse a favor da inclinação, e menor nos casos em que o vento estivesse contra a inclinação. No entanto, o que se observou foi que a percentagem da floresta ardida e outros parâmetros analisados não seguiram essa lógica.

Uma possível explicação para os resultados contraditórios pode ser a simplificação de alguns elementos do modelo, bem como a aleatoriedade de fatores como a geração de fagulhas. Outro fator que poderá ter contribuído para essa contradição é a alta probabilidade de propagação utilizada, que pode ter mascarado o efeito esperado da direção do vento e da inclinação do terreno.

A partir das conclusões deste trabalho, é possível sugerir melhorias no modelo de simulação para investigação futura. Estas melhorias podem incluir a consideração de outros fatores ambientais, como a humidade do ar e do solo, e uma análise mais aprofundada das probabilidades de propagação. Além disso, a realização de mais testes com diferentes configurações de parâmetros pode ajudar a compreender melhor as discrepâncias observadas.

Concluímos que a simulação baseada em agentes é uma ferramenta valiosa no estudo de incêndios florestais, permitindo aos investigadores compreender melhor as complexas interações e comportamentos envolvidos. Essas simulações têm sido usadas para explorar tópicos como estratégias de combate a incêndios, impacto do clima e topografia, e fatores humanos na gestão de incêndios. A melhor compreensão destes fatores pode levar a decisões mais informadas e práticas mais eficazes, tornando a simulação baseada em agentes uma ferramenta essencial na redução do impacto dos incêndios florestais. Embora os resultados não tenham sido os esperados, o trabalho realizado neste projeto servirá como base para futuras investigações e melhorias no campo da prevenção e combate a incêndios florestais.